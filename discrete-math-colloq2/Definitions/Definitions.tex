\documentclass[a4paper,12pt]{article}

\usepackage{header}

\begin{document}
	\title{Дискретная математика. Коллоквиум весна 2017.\\ Определения}
	\author{Потом заполню}
	\maketitle
	
	%Определения даются просто как \item.
	\begin{enumerate}
	
	%1
	 \item 
    
	Пространством элементарных исходов   ${\mathbf \Omega}$ («омега») называется конечное множество, содержащее все возможные результаты 	данного случайного эксперимента, из которых в эксперименте происходит ровно один. Элементы этого множества называют элементарными исходами  и обозначают буквой $\omega$ («омега») с индексами или без.

	Событиями  мы будем называть подмножества множества ${\mathbf \Omega}$. Говорят, что в результате эксперимента произошло событие  	$A\subseteq \mathbf \Omega$, если в эксперименте произошел один из элементарных исходов, входящих в множество $A$.

	Поставим каждому элементарному исходу  $\omega_i\in \mathbf\Omega$ в соответствие число  $p (\omega_i)\in [0,1]$ так, что

	\[\sum_{\omega_i\in \mathbf\Omega} p(\omega_i) =1.\]

	Назовем число $p (\omega_i)$ вероятностью  элементарного исхода $\omega_i$. Вероятностью  события $A~\subseteq~\mathbf\Omega$ 	называется число

	$${\mathsf P}(A) = \sum_{\omega_i\in A} p(\omega_i),$$
	равное сумме вероятностей элементарных исходов, входящих в множество $A$.
	\item 
    
    Случайный граф на $n$ вершинах --- элемент вероятностного пространства $\Omega$, состоящего из всевозможных графов на $n$ вершинах, каждому из которых приписана некоторая вероятность. В терминалогии данного курса, граф не содержит петель и кратных ребер, поэтому всего графов на $n$ вершинах $2^{n \choose 2}$ $\Rightarrow |\Omega| = 2^{n \choose 2}$. Понятно, что случайным будет множество ребер графа.
    
    Пример конструкции --- каждому графу $\Omega$ присвоена одинаковая вероятность, т.е. все графы равно вероятны (т.е. для любого графа $G = (V, E) \in \Omega$, для каждой пары вершин $u, v \in V$, $Pr[(u, v) \in E] = \frac{1}{2}$).
    
    Случайные графы используются для изучения каких-то свойств графов. Например, нестрогая постановка вопроса при работе со случайными графами: велика ли вероятность того, что граф обладает данным свойством? Более конкретный пример использования: доказательство того, что при достаточно большом числе вершин, случайный граф (в равновозможной модели) будет почти всегда связен. Формально: $\Omega_n$ --- вероятностное пространство состоящее из графов на $n$ вершинах, все графы равновозможны, событие $A_n$ --- случайный граф на $n$ вершинах связен; доказать $ \lim_{n\to\infty} Pr[A_n] = 1 $.
    
	%3
	\item 
    
    Условная вероятность — вероятность наступления одного события при условии, что другое событие уже произошло.
    
   \[
       P(A|B) = \frac{{P(A\cap B)}}{P(B)}
   \]
    
    
    $P(A|B)$ – условная вероятность итога А; 
    
   $P(A\cap B)$ – вероятность совместного появления событий А и В; 
    
    $P(B)$ – вероятность события В. 
    
        \item
        %4
        События $A$ и $B$ называются независимыми, если вероятность композиции событий $P(AB)$ равна произведению вероятностей 		$P(A)\cdot P(B)$


        Свойства независимых событий: 
        \begin{enumerate}
            \item Если $p(B)\ne0$, то условная вероятности $p(A\cap B)$ равна вероятности события $p(A)$.
            \item Если события $A$ и $B$ независимы, то события $\overline{A}$ и $B$, $A$ и $\overline{B}$ и $\overline{A}$ и 	$\overline{B}$ также независимы.
        \end{enumerate}
	
	%5
	\item  
    
	    \textit{Случайная величина} -- функция $f : U \rightarrow \R$ из вероятностного пространства $U$.

	    \textit{Матожидание} случайной величины -- произведение значений случайной величины на соответствующие вероятности. Говоря простым языком, это среднееожидаемое значение при многократном повторении испытаний. Пусть случайная величина Х принимает значения $x_1, x_2 ... x_n$ с вероятностями $p_1, p_2 \ldots p_n$ соответственно. Тогда математическое ожидание  данной случайной величины равно сумме произведений всех её значений на соответствующие вероятности:
    	$$\E(x) = \sum_{i = 1}^n x_ip_i$$
    
		\item 
		Множества называются \textit{равномощными}, если между ними существует \textit{биекция}, или взаимо-однозначное 	соответствие. Равномощность множеств обозначают значком $\sim$.
		
		Свойства равномощности:
		\begin{enumerate}
			\item \textit{Cимметричтность:} $A \sim B \Rightarrow B \sim A$.
			\item \textit{Рефлексивность:} $\forall A: \ A \sim A$
			\item \textit{Транзитивность:} $A \sim B, \ B \sim C  \Rightarrow A \sim C$
		\end{enumerate}
	%7
	\item 
    Бесконечное множество называется \textit{счетным}, если оно равномощно множеству $\N$.

    Примеры: 
    
    \begin{enumerate}
        \item Натуральные числа
        \item Целые числа
        \item Рациональные числа
    \end{enumerate}

    Примером несчетных множеств могут являться следующие множества:
    \begin{enumerate}
        \item Вещественные числа
        \item Комплексные числа
    \end{enumerate}

	\item 
	Счетные множества обладают некоторыми свойствами:
	\begin{enumerate}
        \item Всякое подмножество счетного множества конечно или счетно
        \item Конечное либо счетное объединение конечных либо счетных множеств конечно либо счетно.
		\item Объединение счетных множеств счетно
		\item Всякое бесконечное множество содержит счетное подмножество
		\item Множество $\Q$ рациональных чисел счетно
		
		\item Декартово произведение счетных множеств $A \times B$ счетно.
		\item Число слов в конечном или счетном алфавите счетно.
	\end{enumerate}
	\item 
	Континуум - мощность множества [0,1]. Примеры:
	\begin{enumerate}
		\item Множество бесконечных последовательностей нулей и единиц
		\item Множество вещественных чисел
		\item Квадрат [0,1]x[0,1].
	\end{enumerate}
	\item 
	Свойства континуума:
	\begin{enumerate}
	\item В любом континуальном множестве есть счетное подмножество.
	\item Мощность объединения не более чем континуального семейства множеств,
	каждое из которых не более чем континуально, не превосходит континуума.
    \item Если континуальное множество представимо в виде счетного объединения его подмножеств, то по крайней мере одно подмножество должно быть континуальнм.
	\end{enumerate}
	
	\item
	Булева функция от $n$ аргументов - отображение из $B^{n}$ в $B$, где B - $\{0,1\}$.
	Количество всех $n$-арных булевых функций равно $2^{2^{n}}$. Булеву функцию можно задать таблицей истинности.
	
	\item 
	Полный базис - это такой набор, который для реализации любой сколь
	угодно сложной логической функции не потребует использования каких-либо других
	операций, не входящих в этот набор.
	Примеры полных базисов:
    
	\begin {enumerate}
	\item Конъюнкция, дизъюнкция, отрицание.
	\item Конъюнкция, отрицание.
	\item Конъюнкция, сложение по модулю два, константа один - базис Жегалкина.
	\item Штрих Шеффера (таблица истинности - 0111).
	\end {enumerate}
    
    \item \textit{Разложением Шеннона} функции $f: \{0, 1\}^n \rightarrow \{0, 1\}$ по переменной $x_i$ называется представление фукнции $f$ в виде:
    \[
    f(x_n, \ldots, x_i, \ldots, x_1) = \overline {x_i} \cdot f(x_n, \ldots, 0, \ldots x_1) \lor x_i \cdot f(x_n, \ldots, 1, \ldots, x_1)
    \]
    
    \textit{Разложением Рида} называется следующее представление функции:
    \[
    f(x_n, \ldots, x_i, \ldots, x_1) = g_0 \oplus (g_0 \oplus g_1) \cdot x_i),
    \]
    \[
    g_0 = f(x_n, \ldots, 0, \ldots x_1)
    \]
    \[
    g_1 = f(x_n, \ldots, 1, \ldots, x_1)
    \]
    \setcounter{enumi}{15}
    \item 
    \textit{Булевой схемой} от $n$ переменных $x_1, \ldots, x_n$ называется последовательность булевых функций $g_1, \ldots, g_s$, в которой всякая $g_i$ или равна одной из переменных,
    или получается из предыдущих применением одной из логических операций из \textit{базиса схемы}. Также в булевой схеме задано некоторое число $m \geq 1$
    и члены последовательности $g_{s-m+1}, \ldots, g_s$ называются выходами схемы.
    Число m называют числом выходов. Число s называют размером схемы.
    
    \setcounter{enumi}{17}
	\item
    %18
    Свойства вычислимой функции:
    \begin{enumerate}
        \item Если функция $f$ вычислима, то её область определения $D(f)$ является перечислимым множеством.
        \item Если функция $f$ вычислима, то её область значений $E(f)$ является перечислимым множеством.
        \item Если функция $f$ вычислима, то для любого перечислимого множества $X$ его образ $f(X)$ является перечислимым множеством.
        \item Если функция $f$ вычислима, то для любого перечислимого множества $X$ его прообраз $f^{-1}(X)$ является перечислимым множеством.
        \item Композиция вычислимых функция также является вычислимой
    \end{enumerate}
    \item
    %19
    Множество называется \textit{разрешимым}, если для него существует разрешающий алгоритм, который на любом входе останавливается за конечное число шагов ({\itshape разрешающий алгоритм для множества --- алгоритм, получающий на вход натуральное число и определяющий, принадлежит ли оно данному множеству}). Говорят, что такой алгоритм вычисляет \textit{характеристическую функцию}
    \item
    %20
    Счетное множество называется \textit{перечислимым}, если все его элементы могут быть получены с помощью некоторого алгоритма.
    \item
    %21
    Свойства перечислимых множеств:
    \begin{enumerate}
        \item Если множества $A$ и $B$ перечислимы, то их объединение $A \cup B$ и пересечение $A \cap B$ также перечислимы (\textit{отсюда следует, что объединение или пересечение конечного числа перечислимых множеств перечислимо}).
        \item Если множество $A$ перечислимо, то оно является областью значений некоторой вычислимой функции (\textit{это также является достаточным условием перечислимости}).
        \item Если множество $A$ перечислимо, то оно является областью определения некоторой вычислимой функции (\textit{это также является достаточным условием перечислимости}).
    \end{enumerate}
    \item
    %22
    Функция $U:\N\times\N\to\N$ называется универсальной, если для любой функции $f: \N\to\N$ существует такое $p$, что $U(p, x)=f(x)$ для любых $x$ (\textit{равенство здесь понимается в том смысле, что при любом $x$ обе функции либо принимают одинаковое значение, либо не определены}).
    \item
	\item 
	Универсальную вычислимую функцию $U(p, x)$ называют \textit{главной}, если для любой вычислимой функции $V(q, y)$ существует \textit{транслятор} -- вычислимая тотальная функция, такая что 
	\[
	\forall q, y: \ V(q, y) = U(s(q), y)
	\]
	Такие функции также называются \textit{главными нумерациями}.
	\item 
	Пусть $F = \{ f \ | \ f : \N \rightarrow \N \}$ - множество вычислимых функций. Пусть $A \subseteq F$ - подмножество функций. Говорят, что функция удовлетворяет некому \textit{свойству}, если она лежит в $A$. Пусть $U(p, x)$ -- универсальная функция. Пусть $P_a = \{ p \ | \ U(p, x) \in A \}$. Утверждается, что если A -- нетривиально (т.е $A \neq \oslash, \overline{A} \neq \oslash$), то множество $P_a$ неразрешимо.
   	\item 
    Пусть $U(p, x)$ -- главная нумерация. Тогда для любой тотальной вычислимой функции $p(t) \  \exists t: \ U(p(t), x) = U(t, x)$. Это утверждение называется теоремой о неподвижной точке.
   \item
   \textit{Машиной Тьюринга (МТ)} называется модель вычислений, состоящая из
   \begin{enumerate}
       \item бесконечной в обе стороны \textit{ленты}, в которой могут быть записаны символы \textit{конечного алфавита} $A$;
       \item \textit{головки}, которая может двигаться по ленте и которая может работать в один момент времени только с одной ячейкой;
       \item \textit{множества состояний} $Q$;
       \item \textit{таблицы переходов}, которая задает функцию 
       \[
       \delta: \ A \times Q \rightarrow A \times Q \times \{-1, 0, +1\}
       \]
   \end{enumerate}
   МТ может быть \textit{многоленточной}, тогда число головок будет соответствовать числу лент, и наша функция $\delta$ на МТ с $n$ лент примет вид
   \[
   \delta: \ A^n \times Q \rightarrow A^n \times Q \times \{-1, 0, +1\}^n
   \]
   
   \item Функция $f: B* \rightarrow B$, где $B$ -- подмножество алфавита машины без пустого символа, вычислима на МТ, если для любого $x$ из области определения функции результат работы МТ равен $f(w)$, иначе МТ не останавливается. Иными словами, фукнция вычислима на МТ, если для любого входа, на котором функция определена, МТ останавливается и выдает правильный ответ, иначе она не останавливается.	
\end{enumerate}
		
	
\end{document}
